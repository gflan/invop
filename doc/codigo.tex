\section{Código fuente}

Se omite el launcher del lexer dado que solamente es llamar al lexer desde un main y no forma parte del programa principal.

\subsection{tokrules.py}

\definecolor{codegreen}{rgb}{0,0.6,0}
\definecolor{codegray}{rgb}{0.5,0.5,0.5}
\definecolor{codepurple}{rgb}{0.58,0,0.82}
\definecolor{backcolour}{rgb}{0.95,0.95,0.92}

\definecolor{codegreen}{rgb}{0,0.6,0}
\definecolor{codegray}{rgb}{0.5,0.5,0.5}
\definecolor{codepurple}{rgb}{0.58,0,0.82}
\definecolor{backcolour}{rgb}{0.95,0.95,0.92}

\lstdefinestyle{mystyle}{
    backgroundcolor=\color{backcolour},
    commentstyle=\color{codegreen},
    keywordstyle=\color{magenta},
    numberstyle=\tiny\color{codegray},
    stringstyle=\color{codepurple},
    basicstyle=\footnotesize,
    breakatwhitespace=false,
    breaklines=true,
    captionpos=b,
    keepspaces=true,
    numbers=left,
    numbersep=5pt,
    showspaces=false,
    showstringspaces=false,
    showtabs=false,
    tabsize=2
}

\lstset{style=mystyle}
    \begin{lstlisting}[language=Python]
        import ply.lex as lex

        tokens = (
            'CHR',
            'DIVIDE'
        )

        # los literales son chars que matchean de una
        literals = "_^(){}"

        def t_CHR(t):
            # el primer ^ toma complemento de los simbolos que siguen
            r'[^ _ \^ / \( \)\{\}]'
            return t

        def t_DIVIDE(t):
            r'/'
            return t

        t_ignore  = '\t'

        def t_error(t):
            print("Illegal character '%s'" % t.value[0])
            t.lexer.skip(1)

    \end{lstlisting}

\subsection{AST.py}

    \begin{lstlisting}[language=Python]
        class Expr: pass




    \end{lstlisting}

\subsection{parser\_rules.py}

    \begin{lstlisting}[language=Python]
        import ply.yacc as yacc
        from tokrules import tokens
        from AST import *

        SYNTAX_ERROR_IN_INPUT_ERROR_MESSAGE = "Syntax error in input!"


    \end{lstlisting}


\subsection{AST\_visitors.py}

    \begin{lstlisting}[language=Python]
        from AST import *
        from copy import copy

            def visitGroupedPar(self, grouped_expr):
                grouped_expr.expr.accept(self)
                times_scale = grouped_expr.h/grouped_expr.e
                grouped_expr.svg = "<text x=\"0\" y=\"0\" font-size=\"{}\" transform=\"translate({},{}) scale(1,{})\">(</text> \n".format(grouped_expr.e, grouped_expr.x, grouped_expr.y-(times_scale-1)*0.2*grouped_expr.e, times_scale)

                grouped_expr.svg += grouped_expr.expr.svg

                grouped_expr.svg += "<text x=\"0\" y=\"0\" font-size=\"{}\" transform=\"translate({},{}) scale(1,{})\">)</text> \n".format(grouped_expr.e, grouped_expr.x + grouped_expr.a - 0.6*grouped_expr.e, grouped_expr.y-(times_scale-1)*0.2*grouped_expr.e, times_scale)

    \end{lstlisting}


\subsection{ParserTest.py}

    \begin{lstlisting}[language=Python]
        import unittest
        import main
        from AST import *
        from AST_visitors import *


                self.assert_equal_ast(ast, "(A^BC^D/E^F_G+H)-I")

            def test_complex_formula_equal_formula_with_curly_brackets(self):
                astComplexFormula = main.ast_generate("(A^BC^D/E^F_G+H)-I")
                curlyBracketsFormula = "{({{A^B}{C^D}}/{{{E^F_G}+}H})-}I"
                astCurlyBrackets = main.ast_generate(curlyBracketsFormula)
                self.assertEqual(astComplexFormula, astCurlyBrackets)

        if __name__ == '__main__':
            unittest.main()

    \end{lstlisting}


    \subsection{main.py}

        \begin{lstlisting}[language=Python]
            #!/usr/bin/python3
        version=\"1.1\" style=\"background: white\">\n<g transform=\"scale(40) translate(1,1)\" font-family=\"Courier\">\n".format(ast_with_attributes.a*50+80, (ast_with_attributes.h1+ast_with_attributes.h2)*40+80, ast_with_attributes.h1)
                result += ast_with_attributes.svg
                result += "</g>\n</svg>\n"

                output_file = open(args.output_filename, "w")
                output_file.write(result)
                output_file.close()

                if args.print_svg:
                    print(result)

        \end{lstlisting}
