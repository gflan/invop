\section{Introducción}

\quad    El presente trabajo práctico consiste en modelar problemas de programación lineal, mixta y entera para luego implementar dichos modelos en CPLEX.

\quad    En particular todas las implementaciones están hechas sobre la \emph{API de Python} de CPLEX. Por cada ejercicio se desarrolla un código \emph{.py} que contiene los llamados al solver de CPLEX y, a excepción del ejercicio 12.06 que se definen las restricciones y función objetivo a través de un archivo \emph{.lp}, la generación de filas y funciones objetivo.

\quad    Por último se prueban distintos tipos de parámetros de CPLEX que afectan la performance según las características del problema. Estos son, por ejemplo, selección de ramificaciones, variables para ramificar, nodos para el backtracking de dichas ramificaciones, uso de presolves y agresividad en la aplicación de cortes.
